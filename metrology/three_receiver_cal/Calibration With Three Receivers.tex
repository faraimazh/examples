
% Default to the notebook output style

    


% Inherit from the specified cell style.




    
\documentclass{article}

    
    
    \usepackage{graphicx} % Used to insert images
    \usepackage{adjustbox} % Used to constrain images to a maximum size 
    \usepackage{color} % Allow colors to be defined
    \usepackage{enumerate} % Needed for markdown enumerations to work
    \usepackage{geometry} % Used to adjust the document margins
    \usepackage{amsmath} % Equations
    \usepackage{amssymb} % Equations
    \usepackage[mathletters]{ucs} % Extended unicode (utf-8) support
    \usepackage[utf8x]{inputenc} % Allow utf-8 characters in the tex document
    \usepackage{fancyvrb} % verbatim replacement that allows latex
    \usepackage{grffile} % extends the file name processing of package graphics 
                         % to support a larger range 
    % The hyperref package gives us a pdf with properly built
    % internal navigation ('pdf bookmarks' for the table of contents,
    % internal cross-reference links, web links for URLs, etc.)
    \usepackage{hyperref}
    \usepackage{longtable} % longtable support required by pandoc >1.10
    \usepackage{booktabs}  % table support for pandoc > 1.12.2
    

    
    
    \definecolor{orange}{cmyk}{0,0.4,0.8,0.2}
    \definecolor{darkorange}{rgb}{.71,0.21,0.01}
    \definecolor{darkgreen}{rgb}{.12,.54,.11}
    \definecolor{myteal}{rgb}{.26, .44, .56}
    \definecolor{gray}{gray}{0.45}
    \definecolor{lightgray}{gray}{.95}
    \definecolor{mediumgray}{gray}{.8}
    \definecolor{inputbackground}{rgb}{.95, .95, .85}
    \definecolor{outputbackground}{rgb}{.95, .95, .95}
    \definecolor{traceback}{rgb}{1, .95, .95}
    % ansi colors
    \definecolor{red}{rgb}{.6,0,0}
    \definecolor{green}{rgb}{0,.65,0}
    \definecolor{brown}{rgb}{0.6,0.6,0}
    \definecolor{blue}{rgb}{0,.145,.698}
    \definecolor{purple}{rgb}{.698,.145,.698}
    \definecolor{cyan}{rgb}{0,.698,.698}
    \definecolor{lightgray}{gray}{0.5}
    
    % bright ansi colors
    \definecolor{darkgray}{gray}{0.25}
    \definecolor{lightred}{rgb}{1.0,0.39,0.28}
    \definecolor{lightgreen}{rgb}{0.48,0.99,0.0}
    \definecolor{lightblue}{rgb}{0.53,0.81,0.92}
    \definecolor{lightpurple}{rgb}{0.87,0.63,0.87}
    \definecolor{lightcyan}{rgb}{0.5,1.0,0.83}
    
    % commands and environments needed by pandoc snippets
    % extracted from the output of `pandoc -s`
    \DefineVerbatimEnvironment{Highlighting}{Verbatim}{commandchars=\\\{\}}
    % Add ',fontsize=\small' for more characters per line
    \newenvironment{Shaded}{}{}
    \newcommand{\KeywordTok}[1]{\textcolor[rgb]{0.00,0.44,0.13}{\textbf{{#1}}}}
    \newcommand{\DataTypeTok}[1]{\textcolor[rgb]{0.56,0.13,0.00}{{#1}}}
    \newcommand{\DecValTok}[1]{\textcolor[rgb]{0.25,0.63,0.44}{{#1}}}
    \newcommand{\BaseNTok}[1]{\textcolor[rgb]{0.25,0.63,0.44}{{#1}}}
    \newcommand{\FloatTok}[1]{\textcolor[rgb]{0.25,0.63,0.44}{{#1}}}
    \newcommand{\CharTok}[1]{\textcolor[rgb]{0.25,0.44,0.63}{{#1}}}
    \newcommand{\StringTok}[1]{\textcolor[rgb]{0.25,0.44,0.63}{{#1}}}
    \newcommand{\CommentTok}[1]{\textcolor[rgb]{0.38,0.63,0.69}{\textit{{#1}}}}
    \newcommand{\OtherTok}[1]{\textcolor[rgb]{0.00,0.44,0.13}{{#1}}}
    \newcommand{\AlertTok}[1]{\textcolor[rgb]{1.00,0.00,0.00}{\textbf{{#1}}}}
    \newcommand{\FunctionTok}[1]{\textcolor[rgb]{0.02,0.16,0.49}{{#1}}}
    \newcommand{\RegionMarkerTok}[1]{{#1}}
    \newcommand{\ErrorTok}[1]{\textcolor[rgb]{1.00,0.00,0.00}{\textbf{{#1}}}}
    \newcommand{\NormalTok}[1]{{#1}}
    
    % Define a nice break command that doesn't care if a line doesn't already
    % exist.
    \def\br{\hspace*{\fill} \\* }
    % Math Jax compatability definitions
    \def\gt{>}
    \def\lt{<}
    % Document parameters
    \title{Calibration With Three Receivers}
    
    
    

    % Pygments definitions
    
\makeatletter
\def\PY@reset{\let\PY@it=\relax \let\PY@bf=\relax%
    \let\PY@ul=\relax \let\PY@tc=\relax%
    \let\PY@bc=\relax \let\PY@ff=\relax}
\def\PY@tok#1{\csname PY@tok@#1\endcsname}
\def\PY@toks#1+{\ifx\relax#1\empty\else%
    \PY@tok{#1}\expandafter\PY@toks\fi}
\def\PY@do#1{\PY@bc{\PY@tc{\PY@ul{%
    \PY@it{\PY@bf{\PY@ff{#1}}}}}}}
\def\PY#1#2{\PY@reset\PY@toks#1+\relax+\PY@do{#2}}

\expandafter\def\csname PY@tok@gd\endcsname{\def\PY@tc##1{\textcolor[rgb]{0.63,0.00,0.00}{##1}}}
\expandafter\def\csname PY@tok@gu\endcsname{\let\PY@bf=\textbf\def\PY@tc##1{\textcolor[rgb]{0.50,0.00,0.50}{##1}}}
\expandafter\def\csname PY@tok@gt\endcsname{\def\PY@tc##1{\textcolor[rgb]{0.00,0.27,0.87}{##1}}}
\expandafter\def\csname PY@tok@gs\endcsname{\let\PY@bf=\textbf}
\expandafter\def\csname PY@tok@gr\endcsname{\def\PY@tc##1{\textcolor[rgb]{1.00,0.00,0.00}{##1}}}
\expandafter\def\csname PY@tok@cm\endcsname{\let\PY@it=\textit\def\PY@tc##1{\textcolor[rgb]{0.25,0.50,0.50}{##1}}}
\expandafter\def\csname PY@tok@vg\endcsname{\def\PY@tc##1{\textcolor[rgb]{0.10,0.09,0.49}{##1}}}
\expandafter\def\csname PY@tok@m\endcsname{\def\PY@tc##1{\textcolor[rgb]{0.40,0.40,0.40}{##1}}}
\expandafter\def\csname PY@tok@mh\endcsname{\def\PY@tc##1{\textcolor[rgb]{0.40,0.40,0.40}{##1}}}
\expandafter\def\csname PY@tok@go\endcsname{\def\PY@tc##1{\textcolor[rgb]{0.53,0.53,0.53}{##1}}}
\expandafter\def\csname PY@tok@ge\endcsname{\let\PY@it=\textit}
\expandafter\def\csname PY@tok@vc\endcsname{\def\PY@tc##1{\textcolor[rgb]{0.10,0.09,0.49}{##1}}}
\expandafter\def\csname PY@tok@il\endcsname{\def\PY@tc##1{\textcolor[rgb]{0.40,0.40,0.40}{##1}}}
\expandafter\def\csname PY@tok@cs\endcsname{\let\PY@it=\textit\def\PY@tc##1{\textcolor[rgb]{0.25,0.50,0.50}{##1}}}
\expandafter\def\csname PY@tok@cp\endcsname{\def\PY@tc##1{\textcolor[rgb]{0.74,0.48,0.00}{##1}}}
\expandafter\def\csname PY@tok@gi\endcsname{\def\PY@tc##1{\textcolor[rgb]{0.00,0.63,0.00}{##1}}}
\expandafter\def\csname PY@tok@gh\endcsname{\let\PY@bf=\textbf\def\PY@tc##1{\textcolor[rgb]{0.00,0.00,0.50}{##1}}}
\expandafter\def\csname PY@tok@ni\endcsname{\let\PY@bf=\textbf\def\PY@tc##1{\textcolor[rgb]{0.60,0.60,0.60}{##1}}}
\expandafter\def\csname PY@tok@nl\endcsname{\def\PY@tc##1{\textcolor[rgb]{0.63,0.63,0.00}{##1}}}
\expandafter\def\csname PY@tok@nn\endcsname{\let\PY@bf=\textbf\def\PY@tc##1{\textcolor[rgb]{0.00,0.00,1.00}{##1}}}
\expandafter\def\csname PY@tok@no\endcsname{\def\PY@tc##1{\textcolor[rgb]{0.53,0.00,0.00}{##1}}}
\expandafter\def\csname PY@tok@na\endcsname{\def\PY@tc##1{\textcolor[rgb]{0.49,0.56,0.16}{##1}}}
\expandafter\def\csname PY@tok@nb\endcsname{\def\PY@tc##1{\textcolor[rgb]{0.00,0.50,0.00}{##1}}}
\expandafter\def\csname PY@tok@nc\endcsname{\let\PY@bf=\textbf\def\PY@tc##1{\textcolor[rgb]{0.00,0.00,1.00}{##1}}}
\expandafter\def\csname PY@tok@nd\endcsname{\def\PY@tc##1{\textcolor[rgb]{0.67,0.13,1.00}{##1}}}
\expandafter\def\csname PY@tok@ne\endcsname{\let\PY@bf=\textbf\def\PY@tc##1{\textcolor[rgb]{0.82,0.25,0.23}{##1}}}
\expandafter\def\csname PY@tok@nf\endcsname{\def\PY@tc##1{\textcolor[rgb]{0.00,0.00,1.00}{##1}}}
\expandafter\def\csname PY@tok@si\endcsname{\let\PY@bf=\textbf\def\PY@tc##1{\textcolor[rgb]{0.73,0.40,0.53}{##1}}}
\expandafter\def\csname PY@tok@s2\endcsname{\def\PY@tc##1{\textcolor[rgb]{0.73,0.13,0.13}{##1}}}
\expandafter\def\csname PY@tok@vi\endcsname{\def\PY@tc##1{\textcolor[rgb]{0.10,0.09,0.49}{##1}}}
\expandafter\def\csname PY@tok@nt\endcsname{\let\PY@bf=\textbf\def\PY@tc##1{\textcolor[rgb]{0.00,0.50,0.00}{##1}}}
\expandafter\def\csname PY@tok@nv\endcsname{\def\PY@tc##1{\textcolor[rgb]{0.10,0.09,0.49}{##1}}}
\expandafter\def\csname PY@tok@s1\endcsname{\def\PY@tc##1{\textcolor[rgb]{0.73,0.13,0.13}{##1}}}
\expandafter\def\csname PY@tok@kd\endcsname{\let\PY@bf=\textbf\def\PY@tc##1{\textcolor[rgb]{0.00,0.50,0.00}{##1}}}
\expandafter\def\csname PY@tok@sh\endcsname{\def\PY@tc##1{\textcolor[rgb]{0.73,0.13,0.13}{##1}}}
\expandafter\def\csname PY@tok@sc\endcsname{\def\PY@tc##1{\textcolor[rgb]{0.73,0.13,0.13}{##1}}}
\expandafter\def\csname PY@tok@sx\endcsname{\def\PY@tc##1{\textcolor[rgb]{0.00,0.50,0.00}{##1}}}
\expandafter\def\csname PY@tok@bp\endcsname{\def\PY@tc##1{\textcolor[rgb]{0.00,0.50,0.00}{##1}}}
\expandafter\def\csname PY@tok@c1\endcsname{\let\PY@it=\textit\def\PY@tc##1{\textcolor[rgb]{0.25,0.50,0.50}{##1}}}
\expandafter\def\csname PY@tok@kc\endcsname{\let\PY@bf=\textbf\def\PY@tc##1{\textcolor[rgb]{0.00,0.50,0.00}{##1}}}
\expandafter\def\csname PY@tok@c\endcsname{\let\PY@it=\textit\def\PY@tc##1{\textcolor[rgb]{0.25,0.50,0.50}{##1}}}
\expandafter\def\csname PY@tok@mf\endcsname{\def\PY@tc##1{\textcolor[rgb]{0.40,0.40,0.40}{##1}}}
\expandafter\def\csname PY@tok@err\endcsname{\def\PY@bc##1{\setlength{\fboxsep}{0pt}\fcolorbox[rgb]{1.00,0.00,0.00}{1,1,1}{\strut ##1}}}
\expandafter\def\csname PY@tok@mb\endcsname{\def\PY@tc##1{\textcolor[rgb]{0.40,0.40,0.40}{##1}}}
\expandafter\def\csname PY@tok@ss\endcsname{\def\PY@tc##1{\textcolor[rgb]{0.10,0.09,0.49}{##1}}}
\expandafter\def\csname PY@tok@sr\endcsname{\def\PY@tc##1{\textcolor[rgb]{0.73,0.40,0.53}{##1}}}
\expandafter\def\csname PY@tok@mo\endcsname{\def\PY@tc##1{\textcolor[rgb]{0.40,0.40,0.40}{##1}}}
\expandafter\def\csname PY@tok@kn\endcsname{\let\PY@bf=\textbf\def\PY@tc##1{\textcolor[rgb]{0.00,0.50,0.00}{##1}}}
\expandafter\def\csname PY@tok@mi\endcsname{\def\PY@tc##1{\textcolor[rgb]{0.40,0.40,0.40}{##1}}}
\expandafter\def\csname PY@tok@gp\endcsname{\let\PY@bf=\textbf\def\PY@tc##1{\textcolor[rgb]{0.00,0.00,0.50}{##1}}}
\expandafter\def\csname PY@tok@o\endcsname{\def\PY@tc##1{\textcolor[rgb]{0.40,0.40,0.40}{##1}}}
\expandafter\def\csname PY@tok@kr\endcsname{\let\PY@bf=\textbf\def\PY@tc##1{\textcolor[rgb]{0.00,0.50,0.00}{##1}}}
\expandafter\def\csname PY@tok@s\endcsname{\def\PY@tc##1{\textcolor[rgb]{0.73,0.13,0.13}{##1}}}
\expandafter\def\csname PY@tok@kp\endcsname{\def\PY@tc##1{\textcolor[rgb]{0.00,0.50,0.00}{##1}}}
\expandafter\def\csname PY@tok@w\endcsname{\def\PY@tc##1{\textcolor[rgb]{0.73,0.73,0.73}{##1}}}
\expandafter\def\csname PY@tok@kt\endcsname{\def\PY@tc##1{\textcolor[rgb]{0.69,0.00,0.25}{##1}}}
\expandafter\def\csname PY@tok@ow\endcsname{\let\PY@bf=\textbf\def\PY@tc##1{\textcolor[rgb]{0.67,0.13,1.00}{##1}}}
\expandafter\def\csname PY@tok@sb\endcsname{\def\PY@tc##1{\textcolor[rgb]{0.73,0.13,0.13}{##1}}}
\expandafter\def\csname PY@tok@k\endcsname{\let\PY@bf=\textbf\def\PY@tc##1{\textcolor[rgb]{0.00,0.50,0.00}{##1}}}
\expandafter\def\csname PY@tok@se\endcsname{\let\PY@bf=\textbf\def\PY@tc##1{\textcolor[rgb]{0.73,0.40,0.13}{##1}}}
\expandafter\def\csname PY@tok@sd\endcsname{\let\PY@it=\textit\def\PY@tc##1{\textcolor[rgb]{0.73,0.13,0.13}{##1}}}

\def\PYZbs{\char`\\}
\def\PYZus{\char`\_}
\def\PYZob{\char`\{}
\def\PYZcb{\char`\}}
\def\PYZca{\char`\^}
\def\PYZam{\char`\&}
\def\PYZlt{\char`\<}
\def\PYZgt{\char`\>}
\def\PYZsh{\char`\#}
\def\PYZpc{\char`\%}
\def\PYZdl{\char`\$}
\def\PYZhy{\char`\-}
\def\PYZsq{\char`\'}
\def\PYZdq{\char`\"}
\def\PYZti{\char`\~}
% for compatibility with earlier versions
\def\PYZat{@}
\def\PYZlb{[}
\def\PYZrb{]}
\makeatother


    % Exact colors from NB
    \definecolor{incolor}{rgb}{0.0, 0.0, 0.5}
    \definecolor{outcolor}{rgb}{0.545, 0.0, 0.0}



    
    % Prevent overflowing lines due to hard-to-break entities
    \sloppy 
    % Setup hyperref package
    \hypersetup{
      breaklinks=true,  % so long urls are correctly broken across lines
      colorlinks=true,
      urlcolor=blue,
      linkcolor=darkorange,
      citecolor=darkgreen,
      }
    % Slightly bigger margins than the latex defaults
    
    \geometry{verbose,tmargin=1in,bmargin=1in,lmargin=1in,rmargin=1in}
    
    

    \begin{document}
    
    
    \maketitle
    
    

    
    \begin{Verbatim}[commandchars=\\\{\}]
{\color{incolor}In [{\color{incolor}30}]:} \PY{k+kn}{from} \PY{n+nn}{IPython.display} \PY{k+kn}{import} \PY{o}{*}
         \PY{o}{\PYZpc{}}\PY{k}{nbtoc}
\end{Verbatim}

    
    
    
    

    \section{Intro}


    It has long been known that full error correction is possible given a
VNA with only three recievers and no internal coaxial switch. However,
since no modern VNA employs such an architecture, the software required
to make fully corrected measurements is not available on today's modern
VNA's.

Recently, the application of Frequency Extender units containing only
three receivers has become more common. Thus, there is a need for full
error correction capability on systems with three receivers and no
internal coaxial switch. This document describes how to use
\href{http://www.scikit-rf.org}{scikit-rf} to fully correct two-port
measurements made on such a system.


    \section{Theory}


    A circuit model for a switch-less three receiver system is shown below.

    \begin{Verbatim}[commandchars=\\\{\}]
{\color{incolor}In [{\color{incolor}31}]:} \PY{n}{SVG}\PY{p}{(}\PY{l+s}{\PYZsq{}}\PY{l+s}{pics/vnaBlockDiagramForwardRotated.svg}\PY{l+s}{\PYZsq{}}\PY{p}{)}
\end{Verbatim}
\texttt{\color{outcolor}Out[{\color{outcolor}31}]:}
    
    \begin{center}
    \adjustimage{max size={0.9\linewidth}{0.9\paperheight}}{Calibration With Three Receivers_files/Calibration With Three Receivers_5_0.pdf}
    \end{center}
    { \hspace*{\fill} \\}
    

    To fully correct an arbitrary two-port, the device must be measured in
two orientations, call these forward and reverse. Because there is no
switch present, this requires the operator to physically flip the
device, and save the pair of measurements. In on-wafer scenarios, one
could fabricate two identical devices, one in each orientation. In
either case, a pair of measurements are required for each DUT before
correction can occur.

While in reality the device is being flipped, one can imaging that the
device is static, and the entire VNA circuitry is flipped. This
interpretation lends itself to implementation, as the existing 12-term
correction can be re-used by simply copying the forward error
coefficients into the corresponding reverse error coefficients. This is
what \texttt{scikit-rf} does internally.


    \section{Worked Example}


    This example demonstrates how to create a
\href{http://scikit-rf.readthedocs.org/en/latest/reference/calibration/generated/skrf.calibration.calibration.TwoPortOnePath.html\#skrf.calibration.calibration.TwoPortOnePath}{TwoPortOnePath}
and
\href{http://scikit-rf.readthedocs.org/en/latest/reference/calibration/generated/skrf.calibration.calibration.EnhancedResponse.html\#skrf.calibration.calibration.EnhancedResponse}{EnhancedResponse}
calibration from measurements taken on a Agilent PNAX with a set of VDI
WR-10 TXRX-RX Frequency Extender heads. Comparisons between the two
algorithms are made by correcting an asymmetric DUT.


    \subsection{Read in the Data}


    The measurements of the calibration standards and DUT's were downloaded
from the VNA by saving touchstone files of the raw s-parameter data to
disk. This currently reside in the folder \texttt{./data/}

    \begin{Verbatim}[commandchars=\\\{\}]
{\color{incolor}In [{\color{incolor}32}]:} \PY{n}{ls} \PY{n}{data}\PY{o}{/}
\end{Verbatim}

    \begin{Verbatim}[commandchars=\\\{\}]
attenuator (forward).s2p      short.s2p
attenuator (reverse).s2p      thru.s2p
load.s2p                      wr15 shim and swg (forward).s2p
quarter wave delay short.s2p  wr15 shim and swg (reverse).s2p
    \end{Verbatim}

    These files can be read by scikit-rf into \texttt{Network}s with the
following.

    \begin{Verbatim}[commandchars=\\\{\}]
{\color{incolor}In [{\color{incolor}33}]:} \PY{k+kn}{import} \PY{n+nn}{skrf} \PY{k+kn}{as} \PY{n+nn}{rf} 
         \PY{n}{raw} \PY{o}{=} \PY{n}{rf}\PY{o}{.}\PY{n}{read\PYZus{}all\PYZus{}networks}\PY{p}{(}\PY{l+s}{\PYZsq{}}\PY{l+s}{data/}\PY{l+s}{\PYZsq{}}\PY{p}{)}
         \PY{c}{\PYZsh{} list the raw measurments }
         \PY{n}{raw}\PY{o}{.}\PY{n}{keys}\PY{p}{(}\PY{p}{)}
\end{Verbatim}

            \begin{Verbatim}[commandchars=\\\{\}]
{\color{outcolor}Out[{\color{outcolor}33}]:} ['load',
          'attenuator (reverse)',
          'short',
          'attenuator (forward)',
          'wr15 shim and swg (reverse)',
          'wr15 shim and swg (forward)',
          'thru',
          'quarter wave delay short']
\end{Verbatim}
        
    Each \texttt{Network} can be accessed through the dictionary
\texttt{raw}.

    \begin{Verbatim}[commandchars=\\\{\}]
{\color{incolor}In [{\color{incolor}34}]:} \PY{n}{thru} \PY{o}{=} \PY{n}{raw}\PY{p}{[}\PY{l+s}{\PYZsq{}}\PY{l+s}{thru}\PY{l+s}{\PYZsq{}}\PY{p}{]}
         \PY{n}{thru}
\end{Verbatim}

            \begin{Verbatim}[commandchars=\\\{\}]
{\color{outcolor}Out[{\color{outcolor}34}]:} 2-Port Network: 'thru',  60-90 GHz, 721 pts, z0=[ 50.+0.j  50.+0.j]
\end{Verbatim}
        
    If we look at the \emph{raw} measurement of the flush thru, it can be
seen that only $S_{11}$ and $S_{21}$ contain meaningful data. The other
s-parameters are noise.

    \begin{Verbatim}[commandchars=\\\{\}]
{\color{incolor}In [{\color{incolor}35}]:} \PY{n}{thru}\PY{o}{.}\PY{n}{plot\PYZus{}s\PYZus{}db}\PY{p}{(}\PY{p}{)}
\end{Verbatim}

    \begin{center}
    \adjustimage{max size={0.9\linewidth}{0.9\paperheight}}{Calibration With Three Receivers_files/Calibration With Three Receivers_17_0.png}
    \end{center}
    { \hspace*{\fill} \\}
    

    \subsection{Create Calibration}


    In the code that follows a TwoPortOnePath calibration is created from
corresponding measured and ideal responses of the calibration standards.
The measured networks are read from disk, while their corresponding
ideal responses are generated using scikit-rf. More information about
using scikit-rf to do offline calibrations can be found
\href{http://scikit-rf.readthedocs.org/en/latest/tutorials/calibration.html}{here}.

    \begin{Verbatim}[commandchars=\\\{\}]
{\color{incolor}In [{\color{incolor}36}]:} \PY{k+kn}{from} \PY{n+nn}{skrf.calibration} \PY{k+kn}{import} \PY{n}{TwoPortOnePath}
         \PY{k+kn}{from} \PY{n+nn}{skrf.media} \PY{k+kn}{import} \PY{n}{RectangularWaveguide}
         \PY{k+kn}{from} \PY{n+nn}{skrf} \PY{k+kn}{import} \PY{n}{two\PYZus{}port\PYZus{}reflect} \PY{k}{as} \PY{n}{tpr}
         \PY{k+kn}{from} \PY{n+nn}{skrf} \PY{k+kn}{import}  \PY{n}{mil}
         
         \PY{c}{\PYZsh{} pull frequency information from measurements}
         \PY{n}{frequency} \PY{o}{=} \PY{n}{raw}\PY{p}{[}\PY{l+s}{\PYZsq{}}\PY{l+s}{short}\PY{l+s}{\PYZsq{}}\PY{p}{]}\PY{o}{.}\PY{n}{frequency}
         
         \PY{c}{\PYZsh{} the media object }
         \PY{n}{wg} \PY{o}{=} \PY{n}{RectangularWaveguide}\PY{p}{(}\PY{n}{frequency}\PY{o}{=}\PY{n}{frequency}\PY{p}{,} \PY{n}{a}\PY{o}{=}\PY{l+m+mi}{120}\PY{o}{*}\PY{n}{mil}\PY{p}{,} \PY{n}{z0}\PY{o}{=}\PY{l+m+mi}{50}\PY{p}{)}
         
         \PY{c}{\PYZsh{} list of \PYZsq{}ideal\PYZsq{} responses of the calibration standards}
         \PY{n}{ideals} \PY{o}{=} \PY{p}{[}\PY{n}{wg}\PY{o}{.}\PY{n}{short}\PY{p}{(}\PY{n}{nports}\PY{o}{=}\PY{l+m+mi}{2}\PY{p}{)}\PY{p}{,}
                   \PY{n}{tpr}\PY{p}{(}\PY{n}{wg}\PY{o}{.}\PY{n}{delay\PYZus{}short}\PY{p}{(} \PY{l+m+mi}{90}\PY{p}{,}\PY{l+s}{\PYZsq{}}\PY{l+s}{deg}\PY{l+s}{\PYZsq{}}\PY{p}{)}\PY{p}{,} \PY{n}{wg}\PY{o}{.}\PY{n}{match}\PY{p}{(}\PY{p}{)}\PY{p}{)}\PY{p}{,}
                   \PY{n}{wg}\PY{o}{.}\PY{n}{match}\PY{p}{(}\PY{n}{nports}\PY{o}{=}\PY{l+m+mi}{2}\PY{p}{)}\PY{p}{,}
                   \PY{n}{wg}\PY{o}{.}\PY{n}{thru}\PY{p}{(}\PY{p}{)}\PY{p}{]}
         
         \PY{c}{\PYZsh{} corresponding measurments to the \PYZsq{}ideals\PYZsq{}}
         \PY{n}{measured} \PY{o}{=} \PY{p}{[}\PY{n}{raw}\PY{p}{[}\PY{l+s}{\PYZsq{}}\PY{l+s}{short}\PY{l+s}{\PYZsq{}}\PY{p}{]}\PY{p}{,}
                     \PY{n}{raw}\PY{p}{[}\PY{l+s}{\PYZsq{}}\PY{l+s}{quarter wave delay short}\PY{l+s}{\PYZsq{}}\PY{p}{]}\PY{p}{,}
                     \PY{n}{raw}\PY{p}{[}\PY{l+s}{\PYZsq{}}\PY{l+s}{load}\PY{l+s}{\PYZsq{}}\PY{p}{]}\PY{p}{,}
                     \PY{n}{raw}\PY{p}{[}\PY{l+s}{\PYZsq{}}\PY{l+s}{thru}\PY{l+s}{\PYZsq{}}\PY{p}{]}\PY{p}{]}
         
         \PY{c}{\PYZsh{} the Calibration object}
         \PY{n}{cal} \PY{o}{=} \PY{n}{TwoPortOnePath}\PY{p}{(}\PY{n}{measured} \PY{o}{=} \PY{n}{measured}\PY{p}{,} \PY{n}{ideals} \PY{o}{=} \PY{n}{ideals} \PY{p}{)}
\end{Verbatim}


    \subsection{Apply Correction}


    There are two types of correction possible with a 3-receiver system.

\begin{enumerate}
\def\labelenumi{\arabic{enumi}.}
\itemsep1pt\parskip0pt\parsep0pt
\item
  Full (TwoPortOnePath)
\item
  Partial (EnhancedResponse)
\end{enumerate}

\texttt{scikit-rf} uses the same \texttt{Calibration} object for both,
but employs different correction algorithms depending on the
\texttt{type} of the DUT.

The DUT this example is a WR-15 shim cascaded with a WR-12 1" straight
waveguide, as shown in the picture below. Measurements of this DUT are
corrected with both \emph{full} and \emph{partial} correction and the
results are compared below.

    \begin{Verbatim}[commandchars=\\\{\}]
{\color{incolor}In [{\color{incolor}37}]:} \PY{n}{Image}\PY{p}{(}\PY{l+s}{\PYZsq{}}\PY{l+s}{pics/asymmetic DUT.jpg}\PY{l+s}{\PYZsq{}}\PY{p}{,} \PY{n}{width}\PY{o}{=}\PY{l+s}{\PYZsq{}}\PY{l+s}{75}\PY{l+s}{\PYZpc{}}\PY{l+s}{\PYZsq{}}\PY{p}{)}
\end{Verbatim}
\texttt{\color{outcolor}Out[{\color{outcolor}37}]:}
    
    \begin{center}
    \adjustimage{max size={0.9\linewidth}{0.9\paperheight}}{Calibration With Three Receivers_files/Calibration With Three Receivers_23_0.jpeg}
    \end{center}
    { \hspace*{\fill} \\}
    


    \subsubsection{Full Correction ( TwoPortOnePath)}


    Full correction is achieved by measuring each device in both
orientations, \textbf{forward} and \textbf{reverse}. Meaning the DUT was
physically removed, flipped, and re-inserted. The resulting pair of
measurements are then passed to the \texttt{apply\_cal()} function as a
tuple. This returns a single corrected response.

    \begin{Verbatim}[commandchars=\\\{\}]
{\color{incolor}In [{\color{incolor}38}]:} \PY{n}{dutf} \PY{o}{=} \PY{n}{raw}\PY{p}{[}\PY{l+s}{\PYZsq{}}\PY{l+s}{wr15 shim and swg (forward)}\PY{l+s}{\PYZsq{}}\PY{p}{]}
         \PY{n}{dutr} \PY{o}{=} \PY{n}{raw}\PY{p}{[}\PY{l+s}{\PYZsq{}}\PY{l+s}{wr15 shim and swg (reverse)}\PY{l+s}{\PYZsq{}}\PY{p}{]}
         
         \PY{n}{corrected\PYZus{}full} \PY{o}{=} \PY{n}{cal}\PY{o}{.}\PY{n}{apply\PYZus{}cal}\PY{p}{(}\PY{p}{(}\PY{n}{dutf}\PY{p}{,} \PY{n}{dutr}\PY{p}{)}\PY{p}{)} \PY{c}{\PYZsh{} note argument is an ordered tuple }
         
         \PY{n}{corrected\PYZus{}full}\PY{o}{.}\PY{n}{name} \PY{o}{=} \PY{n+nb+bp}{None}
         \PY{n}{corrected\PYZus{}full}\PY{o}{.}\PY{n}{plot\PYZus{}s\PYZus{}db}\PY{p}{(}\PY{p}{)}
         \PY{n}{title}\PY{p}{(}\PY{l+s}{\PYZsq{}}\PY{l+s}{Full Correction}\PY{l+s}{\PYZsq{}}\PY{p}{)}
\end{Verbatim}

            \begin{Verbatim}[commandchars=\\\{\}]
{\color{outcolor}Out[{\color{outcolor}38}]:} <matplotlib.text.Text at 0x7f81bde1ef90>
\end{Verbatim}
        
    \begin{center}
    \adjustimage{max size={0.9\linewidth}{0.9\paperheight}}{Calibration With Three Receivers_files/Calibration With Three Receivers_26_1.png}
    \end{center}
    { \hspace*{\fill} \\}
    

    \subsubsection{Partial Correction (Enhanced Response)}


    If you pass a single measurement to the
\texttt{TwoPortOnePath.apply\_cal()} function, then it will use partial
correction. This correction is known as \texttt{EnhancedResponse}.
Depending on the measurment application, this type of correction may be
\emph{good enough}, and perhaps the only choice.

    \begin{Verbatim}[commandchars=\\\{\}]
{\color{incolor}In [{\color{incolor}39}]:} \PY{n}{corrected\PYZus{}partial} \PY{o}{=} \PY{n}{cal}\PY{o}{.}\PY{n}{apply\PYZus{}cal}\PY{p}{(}\PY{n}{dutf}\PY{p}{)} \PY{c}{\PYZsh{} note we pass a single network }
         
         \PY{n}{corrected\PYZus{}partial}\PY{o}{.}\PY{n}{name} \PY{o}{=} \PY{n+nb+bp}{None}
         \PY{n}{corrected\PYZus{}partial}\PY{o}{.}\PY{n}{plot\PYZus{}s\PYZus{}db}\PY{p}{(}\PY{l+m+mi}{0}\PY{p}{,}\PY{l+m+mi}{0}\PY{p}{)}
         \PY{n}{corrected\PYZus{}partial}\PY{o}{.}\PY{n}{plot\PYZus{}s\PYZus{}db}\PY{p}{(}\PY{l+m+mi}{1}\PY{p}{,}\PY{l+m+mi}{0}\PY{p}{)}
         \PY{n}{title}\PY{p}{(}\PY{l+s}{\PYZsq{}}\PY{l+s}{Partial Correction}\PY{l+s}{\PYZsq{}}\PY{p}{)}
\end{Verbatim}

            \begin{Verbatim}[commandchars=\\\{\}]
{\color{outcolor}Out[{\color{outcolor}39}]:} <matplotlib.text.Text at 0x7f81c46bb810>
\end{Verbatim}
        
    \begin{center}
    \adjustimage{max size={0.9\linewidth}{0.9\paperheight}}{Calibration With Three Receivers_files/Calibration With Three Receivers_29_1.png}
    \end{center}
    { \hspace*{\fill} \\}
    
    Note the \emph{partially} corrected measurement only has a partially
filled s-matrix, while the fully correct measurement has a full s-matrix

    \begin{Verbatim}[commandchars=\\\{\}]
{\color{incolor}In [{\color{incolor}40}]:} \PY{n}{corrected\PYZus{}partial}\PY{p}{[}\PY{l+s}{\PYZsq{}}\PY{l+s}{75ghz}\PY{l+s}{\PYZsq{}}\PY{p}{]}\PY{o}{.}\PY{n}{s}
\end{Verbatim}

            \begin{Verbatim}[commandchars=\\\{\}]
{\color{outcolor}Out[{\color{outcolor}40}]:} array([[[ 0.01398785+0.00470094j,  0.00000000+0.j        ],
                 [ 0.22933932-0.96883735j,  0.00000000+0.j        ]]])
\end{Verbatim}
        
    \begin{Verbatim}[commandchars=\\\{\}]
{\color{incolor}In [{\color{incolor}41}]:} \PY{n}{corrected\PYZus{}full}\PY{p}{[}\PY{l+s}{\PYZsq{}}\PY{l+s}{75ghz}\PY{l+s}{\PYZsq{}}\PY{p}{]}\PY{o}{.}\PY{n}{s}
\end{Verbatim}

            \begin{Verbatim}[commandchars=\\\{\}]
{\color{outcolor}Out[{\color{outcolor}41}]:} array([[[ 0.09106062-0.05667315j,  0.21885438-0.96923775j],
                 [ 0.22778341-0.95953481j,  0.05839562+0.08057083j]]])
\end{Verbatim}
        

    \subsection{Direct Comparison}


    Below are direct comparisons of the DUT shown above corrected with
\emph{full} and \emph{partial} algorithms. It shows that the partial
calibration produces a large ripple on the reflect measurements, and
slightly larger ripple on the transmissive measurments.

    \begin{Verbatim}[commandchars=\\\{\}]
{\color{incolor}In [{\color{incolor}42}]:} \PY{n}{f}\PY{p}{,} \PY{n}{ax} \PY{o}{=} \PY{n}{subplots}\PY{p}{(}\PY{l+m+mi}{1}\PY{p}{,}\PY{l+m+mi}{2}\PY{p}{,} \PY{n}{figsize}\PY{o}{=}\PY{p}{(}\PY{l+m+mi}{8}\PY{p}{,}\PY{l+m+mi}{4}\PY{p}{)}\PY{p}{)}
         
         \PY{n}{ax}\PY{p}{[}\PY{l+m+mi}{0}\PY{p}{]}\PY{o}{.}\PY{n}{set\PYZus{}title} \PY{p}{(}\PY{l+s}{\PYZsq{}}\PY{l+s}{\PYZdl{}S\PYZus{}\PYZob{}11\PYZcb{}\PYZdl{}}\PY{l+s}{\PYZsq{}}\PY{p}{)}
         \PY{n}{ax}\PY{p}{[}\PY{l+m+mi}{1}\PY{p}{]}\PY{o}{.}\PY{n}{set\PYZus{}title} \PY{p}{(}\PY{l+s}{\PYZsq{}}\PY{l+s}{\PYZdl{}S\PYZus{}\PYZob{}21\PYZcb{}\PYZdl{}}\PY{l+s}{\PYZsq{}}\PY{p}{)}
         
         \PY{n}{corrected\PYZus{}full}\PY{o}{.}\PY{n}{plot\PYZus{}s\PYZus{}db}\PY{p}{(}\PY{l+m+mi}{0}\PY{p}{,}\PY{l+m+mi}{0}\PY{p}{,} \PY{n}{label}\PY{o}{=}\PY{l+s}{\PYZsq{}}\PY{l+s}{Full Correction}\PY{l+s}{\PYZsq{}}\PY{p}{,} \PY{n}{ax} \PY{o}{=} \PY{n}{ax}\PY{p}{[}\PY{l+m+mi}{0}\PY{p}{]}\PY{p}{)}
         \PY{n}{corrected\PYZus{}full}\PY{o}{.}\PY{n}{plot\PYZus{}s\PYZus{}db}\PY{p}{(}\PY{l+m+mi}{1}\PY{p}{,}\PY{l+m+mi}{0}\PY{p}{,} \PY{n}{label}\PY{o}{=}\PY{l+s}{\PYZsq{}}\PY{l+s}{Full Correction}\PY{l+s}{\PYZsq{}}\PY{p}{,} \PY{n}{ax} \PY{o}{=} \PY{n}{ax}\PY{p}{[}\PY{l+m+mi}{1}\PY{p}{]}\PY{p}{)}
         
         \PY{n}{corrected\PYZus{}partial}\PY{o}{.}\PY{n}{plot\PYZus{}s\PYZus{}db}\PY{p}{(}\PY{l+m+mi}{0}\PY{p}{,}\PY{l+m+mi}{0}\PY{p}{,} \PY{n}{label}\PY{o}{=}\PY{l+s}{\PYZsq{}}\PY{l+s}{Partial Correction}\PY{l+s}{\PYZsq{}}\PY{p}{,}\PY{n}{ax}\PY{o}{=}\PY{n}{ax}\PY{p}{[}\PY{l+m+mi}{0}\PY{p}{]}\PY{p}{)}
         \PY{n}{corrected\PYZus{}partial}\PY{o}{.}\PY{n}{plot\PYZus{}s\PYZus{}db}\PY{p}{(}\PY{l+m+mi}{1}\PY{p}{,}\PY{l+m+mi}{0}\PY{p}{,} \PY{n}{label}\PY{o}{=}\PY{l+s}{\PYZsq{}}\PY{l+s}{Partial Correction}\PY{l+s}{\PYZsq{}}\PY{p}{,}\PY{n}{ax}\PY{o}{=}\PY{n}{ax}\PY{p}{[}\PY{l+m+mi}{1}\PY{p}{]}\PY{p}{)}
         
         
         \PY{n}{tight\PYZus{}layout}\PY{p}{(}\PY{p}{)}
\end{Verbatim}

    \begin{center}
    \adjustimage{max size={0.9\linewidth}{0.9\paperheight}}{Calibration With Three Receivers_files/Calibration With Three Receivers_35_0.png}
    \end{center}
    { \hspace*{\fill} \\}
    

    \subsection{Formating}


    \begin{Verbatim}[commandchars=\\\{\}]
{\color{incolor}In [{\color{incolor}43}]:} \PY{k+kn}{from} \PY{n+nn}{IPython.core.display} \PY{k+kn}{import} \PY{n}{HTML}
         
         
         \PY{k}{def} \PY{n+nf}{css\PYZus{}styling}\PY{p}{(}\PY{p}{)}\PY{p}{:}
             \PY{n}{styles} \PY{o}{=} \PY{n+nb}{open}\PY{p}{(}\PY{l+s}{\PYZdq{}}\PY{l+s}{../styles/plotly.css}\PY{l+s}{\PYZdq{}}\PY{p}{,} \PY{l+s}{\PYZdq{}}\PY{l+s}{r}\PY{l+s}{\PYZdq{}}\PY{p}{)}\PY{o}{.}\PY{n}{read}\PY{p}{(}\PY{p}{)}
             \PY{k}{return} \PY{n}{HTML}\PY{p}{(}\PY{n}{styles}\PY{p}{)}
         \PY{n}{css\PYZus{}styling}\PY{p}{(}\PY{p}{)}
\end{Verbatim}

            \begin{Verbatim}[commandchars=\\\{\}]
{\color{outcolor}Out[{\color{outcolor}43}]:} <IPython.core.display.HTML at 0x7f81c4d4d990>
\end{Verbatim}
        

    % Add a bibliography block to the postdoc
    
    
    
    \end{document}
